\section{Project and Timeline}\label{sec:project}
    \subsection{The CARMA Orion Survey}\label{sec:carma}
    The CARMA Orion Survey will form the basis for my thesis. The Combined Array for Research in Millimeter-wave Astronomy (CARMA) was an interferometer of six 10.4-meter, nine 6.1-meter, and eight 3.5-meter antennas located in the Inyo Mountains of California. CARMA operated at a frequency range of 27 to 265 GHz (1mm to 1cm wavelengths) and an angular resolution of 0.15" in its most extended configuration.
    
    The CARMA Orion Survey mapped Orion A and the North American Nebula with CARMA over the course of two years, from May 2013 to April 2015. The survey used the two most compact configurations, corresponding to an angular resolution of 6". The survey includes tracers of outflows and diffuse gas ($^12$CO and $^13$CO), warm dense gas (CS and C$^18$O), shocks (SO), and cold dense gas (CN) with a velocity resolution of 140 m/s. The CARMA maps will be combined with ongoing observations at large single-dish telescopes. %%%%%%%%%%% 
    
    The unique capability of this survey lies in the combination of interferometric observations from CARMA and single-dish observations from the Nobeyama Radio Observatory 45m telescope (NRO). The single-dish observations allow us to probe the large-scale structure of the cloud, while CARMA resolves individual features in detail. At the distance of Orion A, we have access to spatial scales from 0.01 to 10 pc, the highest dynamic range of any Orion CO study to date. The North American Nebula has been mapped in CO at an order of magnitude coarser resolution. This spatial dynamic range is crucial for detecting and quantifying the feedback effects outlined in \S\ref{sec:feedback}.
    
Within the CARMA Orion collaboration, I am responsible for combination of the CARMA and NRO maps (\S\ref{sec:status}) and the search for wind-blown shells in Orion A (\S\ref{sec:paper1}). I will lead the imaging and the study of feedback in the North American Nebula (\S\ref{sec:paper3}). 

    \subsection{Status of Observations}\label{sec:status}
    As of Spring 2016, the CARMA observations are completed and the data reduced. The NRO observations are ongoing. NRO has covered approximately 70\% of the CARMA coverage in $^{12}$CO and $^{13}$CO. The image combination is ongoing as well. I am following the method of \citet{Koda11} to combine the CARMA and NRO data, but we have been. The North American Nebula was observed by 
    
    
 %   \subsection{Preliminary Orion Maps}\label{sec:maps}
 %   In this section I will show some prelimi
    
    \subsection{Dissertation Papers}\label{sec:papers}
        \subsubsection{Wind-Blown Shells in Orion A}\label{sec:paper1}
        Orion A has been observed in CO by #CITE ALL THE CO SURVEY PAPERS HERE###. The CARMA Orion maps represent the best combination of angular resolution (6"), spectral resolution (0.14 km/s), and coverage (0.01 pc to 10 pc) of any CO survey of Orion A to date. This will allow us to characterize feedback in the cloud more completely than ever before. Moreover, this will be the first systematic search for shells from low to intermediate-mass stellar winds in Orion. 
        
        \subsubsection{Multi-Scale Feedback in Orion A}\label{sec:paper2}
        Feedback ladder
        \subsubsection{Shells and Outflows in North American Nebula}\label{sec:paper3}
        The North American Nebula has been observed with the Purple Mountain Obseratory 12.7m telescope by. We will combine the CARMA data with these maps to arrive at a similar spatial dynamic range as achieved in Orion A.
        \subsubsection{The Rungs of the Feedback Ladder: Orion A and the North American Nebula}\label{sec:paper4}
        In my fourth paper, I will 
\section{Conclusions}\label{sec:conclusions}
    
    


  
  