\section{Project and Timeline}\label{sec:project}
    \subsection{The CARMA Orion Survey}\label{sec:carma}
    The CARMA Orion Survey will form the basis for my thesis. The Combined Array for Research in Millimeter-wave Astronomy (CARMA) located in the Inyo Mountains. The CARMA Orion Survey %%%%%%%%%%% 
    
    The unique capability of this survey lies in the combination of interferometric observations from CARMA and single-dish observations from the Nobeyama Radio Observatory 45m telescope (NRO). The single-dish observations allow us to probe the large-scale structure of the cloud, while CARMA resolves individual features in detail. At the distance of Orion A, we have access to spatial scales from 0.01 to 10 pc, the highest dynamic range of any Orion CO study to date. This is crucial for detecting and quantifying the feedback effects outlined in \S\ref{sec:feedback}.
    \subsection{Status of Observations}\label{sec:status}
    As of Spring 2016, the CARMA observations are completed and the data reduced. The NRO observations are ongoing. NRO has covered approximately 70\% of the CARMA coverage in $^{12}$CO and $^{13}$CO. The image combination 
    \subsection{Preliminary Orion Maps}\label{sec:maps}
    
    
    \subsection{Dissertation Papers}\label{sec:papers}
        \subsubsection{Wind-Blown Shells in Orion A}\label{sec:paper1}
        \subsubsection{Multi-Scale Feedback in Orion A}\label{sec:paper2}
        \subsubsection{Shells and Outflows in North American Nebula}\label{sec:paper3}
        The North American Nebula has been observed with the Purple Mountain Obseratory 12.7m telescope. 
        \subsubsection{The Rungs of the Feedback Ladder: Orion A and the North American Nebula}\label{sec:paper4}
\section{Conclusions}\label{sec:conclusions}
    
    


  
  