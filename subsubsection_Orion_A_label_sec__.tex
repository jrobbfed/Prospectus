\subsubsection{Orion A}\label{sec:orion}
The Orion A molecular cloud is located behind the Orion Nebula, at a distance of 400-450 pc \cite{Menten07}. This cloud represents the nearest site of both low and high-mass star formation. The entire cloud may be an example of triggered star formation, as the Orion-Eridanus superbubble is sweeping through the cloud from the northwest to southeast, corresponding to the overall velocity gradient observed in the cloud \cite{Bally08}. The dominant source of UV heating and ionization is the Orion Nebula Cluster (ONC), which contains the Trapezium (or $\Theta^1$Ori) quartet of OB stars. These stars power the M42 HII region and illuminate the Orion Nebula. The UV radiation field from the ONC appears to sculpt Orion A into cometary structures \cite{Bally08}.

Though the Orion Nebula is spectacular, Orion A is also a rich site of low-mass star formation. South of the ONC, in the low-mass cluster L1641-N, \citet{Stanke_2007} identified several bipolar CO outflows which have enough combined energy and momentum to sustain the turbulence observed in the cluster. Also in L1641-N, \citet{Nakamura_2012} found evidence for concentric parsec-scale shells in CO, which they speculate may be formed by the combined stellar winds of the cluster. My first two papers will focus on Orion A. The first will be a systematic search for and characterization of CO shells in Orion A. The second will be a comparison of the effects of FUV heating, ionization, stellar winds, and bipolar outflows on the molecular cloud.