        \subsubsection{Wind-Blown Shells in Orion A}\label{sec:paper1}
Orion A has been mapped in CO by many previous studies (e.g. \citet{Bally87};~\citet{Wilson05};~\citet{Shimajiri11};~\citet{Buckle12};~\citet{Berne14};~\citet{Nishimura15}). The CARMA Orion maps represent the best combination of angular resolution (6"), spectral resolution (0.14 km/s), and coverage (0.01 pc to 10 pc) of any CO survey of Orion A to date. This will allow us to characterize feedback in the cloud more completely than ever before. Moreover, this is the first systematic search for shells from low to intermediate-mass stellar winds in Orion.
        
I have not completed the combination of NRO and CARMA data (see~\S\ref{sec:status}). Using only the single-dish maps, I present an overview of Orion A and an example of a candidate shell below. Figure~\ref{fig:OrionA_CO} shows integrated intensity of CO in Orion A. The integrated intensity map shows that $^{13}$CO traces the higher density integral-shaped filament while $^{12}$CO is optically thick in the filament and traces more diffuse gas. The point sources shown are all potential source of feedback in the cloud.

The northern-most Ae/Be star (T Ori) is near an arc-like feature identified in $^{12}$CO. Figure~\ref{fig:channelmaps} shows this shell candidate in $^{12}$CO at individual velocities. The compact emission at high velocities supports the idea that this shell is expanding outward. Figure 3 shows a position-velocity (PV) diagram taken through a slice across the shell. Overlain on top of the PV diagram is a shell model from \citet{Arce_2011}. The structure around T Ori qualitatively matches the shell model. The model contains only the redshifted part of the shell in the case where the shell is located on the near side of the cloud.  T Ori is a pre-main sequence A3 star, so it does not ionize its surroundings. Stellar winds from T Ori are a potential driving source for this shell.

The shell candidate near T Ori is one example of the type of feature I will catalog in Orion A. A systematic search for such wind-blown shells will involve cross-matching CO, mid-IR (warm dust emission) and source catalogs to provide a compelling case that the shells are generated by stellar winds. Using the ratio of $^{13}$CO to $^{12}$CO I will correct for the optical depth of $^{12}$CO and measure the mass loss and momentum in the shells \cite{Arce_2011}, comparing this to the amount of turbulence in Orion A. The shells can also be used to constrain the stellar wind mass loss rate of the driving sources. \citet{Offner15} finds that BA star winds must be several times stronger than previously thought to produce the observed shells in Perseus.
