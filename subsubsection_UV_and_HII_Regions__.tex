\subsubsection{UV and HII Regions}\label{sec:uv}
Massive stars emit strongly in the FUV, which heats and ionizes surrounding gas. In a molecular cloud, such heating can sculpt features like cometary clouds and pillars (\citet{1983A&A...117..183R};~\citet{Walawender04}). Intermediate mass stars between 3-8 M$_\odot$ do not emit strongly ionizing radiation fields, but the most massive stars produce HII regions which expand and can disrupt much of a cloud. The prototypical, and closest, HII region is M42. Powered by the Trapezium cluster of OB stars, this HII region has carved out an opening in its progenitor molecular cloud, allowing a rare glimpse of forming stars in optical wavelengths (\citet{Odell93};~\citet{Ricci_2008}). Aside from ionizing and disrupting molecular clouds, HII regions may also trigger star formation as they sweep through a cloud \cite{Deharveng_2010}. An expanding HII region collects gas in a layer between the shock front and ionization front, which can exceed the threshold needed for fragmentation. Alternatively, as the ionization front passes over a molecular core the increased pressure of the ionized surrounding can compress the core and trigger collapse.

All of the preceding feedback mechanisms may be important in the structure and evolution of molecular clouds. I will now provide an introduction to the molecular clouds which will be the focus of my thesis.





  