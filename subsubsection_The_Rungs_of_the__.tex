\subsubsection{The Rungs of the Feedback Ladder: Orion A and the North American Nebula}\label{sec:paper4}
        In my fourth paper, I will compare feedback in the two clouds from my thesis, Orion A and the North American Nebula. These two clouds are similar in many ways. They are both part of larger cloud complexes, and each have both low and high mass star formation. One major difference between the clouds is that the massive stars in Orion A are localized in the Orion Nebula Cluster, while the massive stars in the North American Nebula are distributed throughout the region. Also, the entire Orion A cloud exhibits a cometary shape from possible interaction with an expanding superbubble from past supernovae. Such massive stellar feedback is confined to localized ionization and heating in the North American Nebula. Comparing the two clouds will tell us how feedback operates in distinct environments.