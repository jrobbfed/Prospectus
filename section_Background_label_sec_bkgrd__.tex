\section{Background}\label{sec:bkgrd}

\subsection{Star Formation}\label{sec:sf}
Stars form when clouds of molecular gas exceed the critical density where gravitational contraction overcomes the random gas motions. Below I sketch a general outline of the current understanding of the process by which gas becomes stars, as reviewed by \cite{McKee_2007}, \cite{Draine11}, and \cite{Dunham_2014}. 

Molecular gas in the Milky Way is hierarchical. The largest structures, \textit{giant molecular clouds} (GMCs), condense out of diffuse neutral hydrogen through gravitational instabilities caused by colliding flows or spiral arm overdensities. GMCs are made of molecular \textit{clumps}, the self-gravitating structures of gas that form star clusters. The areas of peak density in clumps are termed molecular \textit{cores}. A core collapses when its internal gas pressure is overpowered by gravity, forming a protostar.

The evolution of protostars is roughly tracked by their infrared spectral energy distribution. As the protostar accretes mass and clears the cold material around it, the SED shifts to hotter bolometric temperatures. This change in the SED is denoted by Classes 0-III, as reviewed by \cite{Andre00}.

Protostars add mass from an accretion disk which in turn is fed by infall from a surrounding envelope of dense gas (Class 0). As a protostar evolves this disk grows and the envelope diminishes (Class I). The accretion disk powers a strong wind which triggers bipolar outflows (see \S\ref{sec:outflow}) of molecular gas around the protostar. The accretion and outflow help to clear the envelope, and the protostar becomes a pre-main sequence star (Class II-III). Pre-main sequence and main sequence stars can affect their host cloud through spherical stellar winds (\S\ref{sec:wind}) and UV ionization and heating (\S\ref{sec:uv}).




