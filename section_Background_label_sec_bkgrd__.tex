\section{Background}\label{sec:bkgrd}

\subsection{Star Formation}\label{sec:sf}
Stars form when clouds of molecular gas exceed the critical density where gravitational contraction overcomes the random gas motions. Below I sketch a general outline of the current understanding of the process by which gas becomes stars, as reviewed by \cite{McKee_2007}, \cite{Draine11}, and \cite{Dunham_2014}. 

Molecular gas in the Milky Way is hierarchical. The largest structures, \textit{giant molecular clouds} (GMCs), condense out of diffuse neutral hydrogen through gravitational instabilities caused by colliding flows or spiral arm overdensities. GMCs are made of molecular \textit{clumps}, the self-gravitating structures of gas that form star clusters. The areas of peak density in clumps are termed molecular \textit{cores}. A core collapses when its internal gas pressure is overpowered by gravity, forming a protostar.

The evolution of protostars is roughly tracked by their infrared spectral energy distribution. As the protostar accretes mass and clears the cold material around it, the SED shifts to hotter bolometric temperatures. This change in the SED is denoted by Classes 0-III, as reviewed by \cite{Andre00}

Protostars add mass from an accretion disk which in turn is fed by infall from a surrounding envelope of dense gas (Class 0). As a protostar evolves this disk grows and the envelope diminishes (Class I). The accretion disk powers a strong wind which triggers bipolar outflows (see \S\ref{sec:outflow}) of molecular gas around the protostar. The accretion and outflow help to clear the envelope, and the protostar becomes a pre-main sequence star (Class II-III). Pre-main sequence and main sequence stars can affect their host cloud through spherical stellar winds (\S\ref{sec:wind}) and UV ionization and heating (\S\ref{sec:uv}).

\subsection{The Low Star Formation Efficiency}\label{sec:sfe}
Only a small fraction of a molecular cloud is converted into stars. On average, the star formation efficiency of galactic molecular clouds is $\sim 5\%$ \cite{McKee_2007}. But if gravity and thermal pressure where the only forces at work, clouds should collapse into stars in a dynamical time. Why are molecular clouds so inefficient at forming stars? Potential causes of this inefficiency are magnetic fields, short cloud lifetimes, and turbulence. 

Magnetic fields can support cores against collapse. The ions in such cores are locked to the magnetic fields, while neutral particles and dust gradually diffuse inward (\citet{Mestel_1956};~\citet{Shu_1983}). This process, called \textit{ambipolar diffusion}, slows the collapse of the core and reduces its star formation efficiency. The observed magnetic fields in many clouds are not strong enough to be the only support against gravitational collapse \cite{Crutcher_2012}.

If molecular clouds only exist for brief amounts of time, they may only have a chance to form a few stars before they are completely disrupted by external (e.g. shear from the galaxy's potential) or internal (e.g. supernovae) forces. \citet{Murray_2011} argues that molecular clouds only live for two to three free-fall times, and \citet{Dobbs_2013} use hydrodynamic simulations of a Milky Way-like galaxy to support the claim that molecular clouds are transient. If clouds are not long lived, this removes the necessity for them to be supported against gravitational collapse for long periods of time. 

Molecular clouds exhibit characteristics of supersonic turbulence. \citet{Larson81} showed that the line-width of clouds, clumps, and cores show a power-law dependence on their physical size. The larger the structure, the higher the velocity dispersion within it. This relationship is characteristic of a turbulent medium. Because of this size dependent velocity dispersion, turbulent pressure prevents large scale collapse of a cloud, while allowing local collapse of cores into protostars \cite{Mac_Low_2004}.

\subsection{Turbulence Regulated Star Formation}\label{sec:turb}

However, turbulence dissipates quickly (on the order of a crossing time). So how is it maintained?


\subsection{How is Turbulence Maintained?}\label{sec:turb-maint}
Turbulence may be maintained by external drivers, such as galactic shear \cite{Dobbs_2013} or by feedback from forming stars. \citet{Nakamura_2007} investigate the effect of molecular outflows from protostars on turbulence in a 1.5 pc clump. They show that the outflows inject enough momentum into the clump to maintain turbulence and support the clump against global collapse. \citet{Moraghan_2013} measure the density distribution in a simulated clouds where outflows maintain turbulence. With such feedback, the density distribution increases above log-normal at low densities, representing the cavities excavated around protostars.

\subsection{Stellar Feedback}\label{sec:feedback}
Stellar feedback can potentially maintain turbulence and reduce the star formation efficiency in molecular clouds. While I will focus on these effects, feedback may also trigger star formation and influence the mass assembly of protostars. This motivates my thesis question: \textit{How does feedback from stars of various masses and evolutionary states affect the structure and evolution of molecular clouds?} In the next sections I introduce the different types of feedback important in molecular clouds.

\subsubsection{Bipolar Outflows}\label{sec:outflow}
Bipolar outflows are ubiquitous around low-mass protostars. Outflows are generated by a strong accretion-driven wind which is collimated into a jet by magnetic fields. This jet shocks the surrounding molecular gas, entraining material in bipolar outflows. Bipolar outflows have been studied extensively in CO emission (e.g. \citet{Plunkett_2013}~;\cite{Plunkett_2015}), and their shocks can be identified in optical wavelengths \cite{Reipurth_2001}. The amount of momentum in bipolar outflows may be able to support turbulence on clump scales (\citet{Frank14}~;\citet{Offner_2014}).

\subsubsection{Spherical Winds}\label{sec:wind}
Spherical stellar winds have not been systematically studied for their effects on molecular clouds. Massive (OB) stars are known to drive expanding shells in the ISM. For example, the 'bubbles' identified in the Spitzer GLIMPSE and MIPSGAL surveys of the galactic plane are thought to be produced by such massive stars (\citet{Churchwell_2006};~\citet{Beaumont14}). Powerful winds combine with an expanding ionization front from the intense UV radiation field sweep up gas and dust. The impact of massive stellar winds on \textit{molecular} gas was investigated for a subset of these bubbles by \cite{Beaumont_2010}. Using CO observations, they found shells of molecular gas around the rims of the bubble, and focused on the potential for triggering star formation in these shells by the 'collect and collapse' of gas. \cite{Sidorin14} investigated one bubble in detail and find several molecular clumps in a ring around the bubble.

Intermediate and low-mass stellar winds have been considered insignificant compared to other forms of feedback, because of their relatively low momenta (\citet{Vink01};~\citet{Smith14a}). However, lower mass stars are much more common than OB stars, so their total impact may be significant. \cite{Arce_2011} identified 12 shells in CO data cubes of the Perseus molecular cloud from the COMPLETE survey. These shells have radii of 0.1-3 pc (smaller than high-mass bubbles). Most of the Perseus shells have candidate powering sources of late B or later spectral type; i.e. non-ionizing low and intermediate-mass stars. The shells all exhibit characteristics of expansion in position-velocity space, and many are associated with circular features in MIPS 24 micron images. The total momentum input of the Perseus shells is comparable to that of bipolar outflows, and their combined energy input rate is comparable to the turbulent dissipation rate of the cloud. Thus wind blown shells and bipolar outflows may be able to maintain turbulence in Perseus. \cite{Nakamura12} found evidence for multiple expanding shells in CO data of the L1641-N cloud. These shells are centered on two low-mass protostellar clusters, evidence that multiple low-mass stellar winds can combine for a more significant effect on their clouds, compared to solitary winds. \textit{My first paper (\S\ref{sec:paper1}) will present results of a search for similar shells in the Orion A molecular cloud.}

\subsubsection{UV and HII Regions}\label{sec:uv}
Massive stars emit strongly in the FUV, which heats and ionizes surrounding gas. In a molecular cloud, such heating can sculpt features like cometary clouds and pillars (\citet{1983A&A...117..183R};~\citet{Walawender04}). Intermediate mass stars between 3-8 M$_\odot$ do not emit strongly ionizing radiation fields, but the most massive stars produce HII regions which expand and can disrupt much of a cloud. The prototypical, and closest, HII region is M42. Powered by the Trapezium cluster of OB stars, this HII region has carved out an opening in its progenitor molecular cloud, allowing a rare glimpse of forming stars in optical wavelengths (\citet{Odell93};~\citet{Ricci_2008}). Aside from ionizing and disrupting molecular clouds, HII regions may also trigger star formation as they sweep through a cloud \cite{Deharveng_2010}. An expanding HII region collects gas in a layer between the shock front and ionization front, which can exceed the threshold needed for fragmentation. Alternatively, as the ionization front passes over a molecular core the increased pressure of the ionized surrounding can compress the core and trigger collapse.

All of the preceding feedback mechanisms may be important in the structure and evolution of molecular clouds. I will now provide an introduction to the molecular clouds which will be the focus of my thesis.

\subsection{Molecular Clouds}\label{sec:clouds}

\subsubsection{Orion A}\label{sec:orion}
The Orion A molecular cloud is located behind the Orion Nebula, at a distance of 400-450 pc \cite{Menten07}. This cloud represents the nearest site of both low and high-mass star formation. The entire cloud may be an example of triggered star formation, as the Orion-Eridanus superbubble is sweeping through the cloud from the northwest to southeast, corresponding to the overall velocity gradient observed in the cloud \cite{Bally08}. The dominant source of UV heating and ionization is the Orion Nebula Cluster (ONC), which contains the Trapezium (or $\Theta^1$Ori) quartet of OB stars. These stars power the M42 HII region and illuminate the Orion Nebula. The UV radiation field from the ONC appears to sculpt Orion A into cometary structures \cite{Bally08}.

Though the Orion Nebula is spectacular, Orion A is also a rich site of low-mass star formation. South of the ONC, in the low-mass cluster L1641-N, \citet{Stanke_2007} identified several bipolar CO outflows which have enough combined energy and momentum to sustain the turbulence observed in the cluster. Also in L1641-N, \citet{Nakamura_2012} found evidence for concentric parsec-scale shells in CO, which they speculate may be formed by the combined stellar winds of the cluster. My first two papers will focus on Orion A. The first will be a systematic search for and characterization of CO shells in Orion A. The second will be a comparison of the effects of FUV heating, ionization, stellar winds, and bipolar outflows on the molecular cloud.

\subsubsection{North American Nebula}\label{sec:nan}
The North American Nebula (NAN) is a region with both high and low-mass star formation located at a distance of $\sim$550 pc \cite{Laugalys06}. CO observations by \citet{Zhang14} indicate that 50,000 M$_\odot$ of molecular gas in front of the W80 HII region. The "North America" shape comes from this background HII region. This HII region is expanding into the molecular cloud, powering an expanding CO bubble \cite{Bally80}. Using deep NIR imaging, \citet{Bally14} report cometary structures and many outflows identified by shock-excited H$_2$ emission. There has been no systematic survey of the impact of feedback on the NAN molecular cloud, and this will from the basis of my third paper (\S\ref{sec:paper3}). A comparison of the feedback budget in Orion A and NAN will be the focus of paper four (\S\ref{sec:paper4}). 
  
  