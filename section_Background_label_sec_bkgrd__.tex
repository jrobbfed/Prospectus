\section{Background}\label{sec:bkgrd}

\subsection{Star Formation}\label{sec:sf}
Stars form when clouds of molecular gas exceed the critical density where gravitational contraction overcomes the random gas motions. Below I sketch a general outline of the current understanding of the process by which gas becomes stars, as reviewed by \cite{McKee_2007}, \cite{Draine11}, and \cite{Dunham_2014}. 

Molecular gas in the Milky Way is hierarchical. The largest structures, \textit{giant molecular clouds} (GMCs), condense out of diffuse neutral hydrogen through gravitational instabilities caused by colliding flows or spiral arm overdensities. GMCs are made of molecular \textit{clumps}, the self-gravitating structures of gas that form star clusters. The areas of peak density in clumps are termed molecular \textit{cores}. A core collapses when its internal gas pressure is overpowered by gravity, forming a protostar.

Protostar

\subsection{The Low Star Formation Efficiency}\label{sec:sfe}

\subsection{Turbulence Regulated Star Formation}\label{sec:turb}

\subsection{How is Turbulence Maintained?}\label{sec:turb-maint}

\subsection{Stellar Feedback}\label{sec:feedback}

\subsubsection{Bipolar Outflows}\label{sec:outflow}

\subsubsection{Spherical Winds}\label{sec:wind}

\subsubsection{UV and HII Regions}\label{sec:uv}

\subsection{Molecular Clouds}\label{sec:clouds}

\subsubsection{Orion A}\label{sec:orion}

\subsubsection{North American Nebula}\label{sec:nan}
  
  
  
  
  