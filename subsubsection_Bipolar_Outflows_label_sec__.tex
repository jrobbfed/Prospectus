\subsubsection{Bipolar Outflows}\label{sec:outflow}
Bipolar outflows are ubiquitous around low-mass protostars. Outflows are generated by a strong accretion-driven wind which is collimated into a jet by magnetic fields. This jet shocks the surrounding molecular gas, entraining material in bipolar outflows. Bipolar outflows have been studied extensively in CO emission (e.g. \citet{Plunkett_2013}~;\cite{Plunkett_2015}), and their shocks can be identified in optical wavelengths \cite{Reipurth_2001}. The amount of momentum in bipolar outflows may be able to support turbulence on clump scales (\citet{Frank14}~;\citet{Offner_2014}).
