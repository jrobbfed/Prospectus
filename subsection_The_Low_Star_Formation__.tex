\subsection{The Low Star Formation Efficiency}\label{sec:sfe}
Only a small fraction of a molecular cloud is converted into stars. On average, the star formation efficiency of galactic molecular clouds is $\sim$5\% \cite{McKee_2007}. But if gravity and thermal pressure are the only forces at work, clouds should collapse into stars in a dynamical time. Why are molecular clouds so inefficient at forming stars? Potential causes of this inefficiency are magnetic fields, short cloud lifetimes, and turbulence. 

Magnetic fields can support cores against collapse. The ions in such cores are locked to the magnetic fields, while neutral particles and dust gradually diffuse inward (\citet{Mestel_1956};~\citet{Shu_1983}). This process, called \textit{ambipolar diffusion}, slows the collapse of the core and reduces its star formation efficiency. The observed magnetic fields in many clouds are not strong enough to be the only support against gravitational collapse \cite{Crutcher_2012}.

If molecular clouds only exist for brief amounts of time, they may only have a chance to form a few stars before they are completely disrupted by external (e.g. shear from the galaxy's potential) or internal (e.g. supernovae) forces. \citet{Murray_2011} argues that molecular clouds only live for two to three free-fall times, and \citet{Dobbs_2013} use hydrodynamic simulations of a Milky Way-like galaxy to support the claim that molecular clouds are transient. If clouds are not long lived, this removes the necessity for them to be supported against gravitational collapse for long periods of time. 

Molecular clouds exhibit characteristics of supersonic turbulence. \citet{Larson81} showed that the line-width of clouds, clumps, and cores show a power-law dependence on their physical size. The larger the structure, the higher the velocity dispersion within it. This relationship is characteristic of a turbulent medium. Because of this size dependent velocity dispersion, turbulent pressure prevents large scale collapse of a cloud, while allowing local collapse of cores into protostars \cite{Mac_Low_2004}.
