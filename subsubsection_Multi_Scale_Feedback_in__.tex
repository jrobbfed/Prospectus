\subsubsection{Multi-Scale Feedback in Orion A}\label{sec:paper2}
        Each feedback mechanism discussed in~\S\ref{sec:feedback} is driven by different mass stars and affects a cloud on different scales. In this paper I will combine the work of our collaborators Yoshito Shimajiri (on bipolar outflows) and John Bally (on massive feedback from the Orion Nebula Cluster) with my own work on stellar winds and FUV heating. I will compare turbulent statistics, like density probability distribution functions and power spectra, in regions of Orion A dominated by each mechanism. I will address the question: How do these mechanisms compare in terms of their momenta, impact on cloud turbulence and star formation efficiency, and possible triggered star formation? This will form the first complete picture of the effects of feedback on a molecular cloud.
