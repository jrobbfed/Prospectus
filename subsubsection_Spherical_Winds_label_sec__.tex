\subsubsection{Spherical Winds}\label{sec:wind}
Spherical stellar winds have not been systematically studied for their effects on molecular clouds. Massive (OB) stars are known to drive expanding shells in the ISM. For example, the 'bubbles' identified in the Spitzer GLIMPSE and MIPSGAL surveys of the galactic plane are thought to be produced by such massive stars (\citet{Churchwell_2006};~\citet{Beaumont14}). Powerful winds combine with an expanding ionization front from the intense UV radiation field sweep up gas and dust. The impact of massive stellar winds on \textit{molecular} gas was investigated for a subset of these bubbles by \cite{Beaumont_2010}. Using CO observations, they found shells of molecular gas around the rims of the bubble, and focused on the potential for triggering star formation in these shells by the 'collect and collapse' of gas. \citet{Sidorin14} investigate one bubble in detail and find several molecular clumps in a ring around the bubble.

Intermediate and low-mass stellar winds have been considered insignificant compared to other forms of feedback, because of their relatively low momenta (\citet{Vink01};~\citet{Smith14a}). However, lower mass stars are much more common than OB stars, so their total impact may be significant. \citet{Arce_2011} identified 12 shells in CO data cubes of the Perseus molecular cloud from the COMPLETE survey. These shells have radii of 0.1-3 pc (smaller than high-mass bubbles). Most of the Perseus shells have candidate powering sources of late B or later spectral type; i.e. non-ionizing low and intermediate-mass stars. The shells all exhibit characteristics of expansion in position-velocity space, and many are associated with circular features in MIPS 24 micron images. The total momentum input of the Perseus shells is comparable to that of bipolar outflows, and their combined energy input rate is comparable to the turbulent dissipation rate of the cloud. Thus wind blown shells and bipolar outflows may be able to maintain turbulence in Perseus. \citet{Nakamura12} found evidence for multiple expanding shells in CO data of the L1641-N cloud. These shells are centered on two low-mass protostellar clusters, evidence that multiple low-mass stellar winds can combine for a more significant effect on their clouds, compared to solitary winds. \textit{My first paper (\S\ref{sec:paper1}) will present results of a search for similar shells in the Orion A molecular cloud.}
