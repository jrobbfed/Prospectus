\section{Background}\label{sec:bkgrd}
\subsection{Star Formation}\label{sec:sf}
\subsection{Stellar Feedback}\label{sec:feedback}
\subsection{Molecular Clouds in This Study}\label{sec:mc}
\subsubsection{Orion A}\label{sec:orion}
\subsubsection{North American Nebula}\label{sec:nan}

\section{Project and Timeline}\label{sec:project}
    \subsection{The CARMA Orion Survey}\label{sec:carma}
    The CARMA Orion Survey will form the basis for my thesis. The Combined Array for Research in Millimeter-wave Astronomy (CARMA) located in the Inyo Mountains . The CARMA Orion Survey 
    
    The unique capability of this survey lies in the combination of interferometric observations from CARMA and single-dish observations from the Nobeyama Radio Observatory 45m telescope. The single-dish allows us to probe large-scale structure, while CARMA has the resolution. At the distance of Orion A, this translates to spatial resolution from 0.01 to 10 pc, the highest dynamic range of any Orion CO study to date. This high dynamic range in spatial scales is crucial for detecting and comparing the feedback effects outlined in \S\ref{sec:feedback}
    \subsection{Preliminary Orion Maps}\label{sec:maps}
    
    \subsection{Dissertation Papers}\label{sec:papers}
        \subsubsection{Wind-Blown Shells in Orion A}\label{sec:paper1}
        \subsubsection{Multi-Scale Feedback in Orion A}\label{sec:paper2}
        \subsubsection{Shells and Outflows in North American Nebula}\label{sec:paper3}
        \subsubsection{The Rungs of the Feedback Ladder: Orion A and the North American Nebula}\label{sec:paper4}
\section{Conclusions}\label{sec:conclusions}
    
    


  
  